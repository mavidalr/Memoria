%!TEX root = ../memoria.tex

\chapter*{Introducción}
\addcontentsline{toc}{chapter}{Introducción}

\markboth{INTRODUCCIÓN}{}


Tras ocho años de discusión en el congreso, el 18 de abril de 2015 se publicó la ley 20810 donde se declara que “(…) las radioemisoras que operen concesiones de radiodifusión sonora, en su programación diaria deberán emitir al menos una quinta parte (20\%) de música nacional, medida sobre el total de canciones emitidas, distribuida durante la jornada diaria de transmisión de cada emisora” [Biblioteca Congreso Nacional de Chile. (2015). Fija porcentajes mínimos de emisión de música nacional y música de raíz folklórica oral, a la radio difusión chilena. 13 de junio, 2016, de Biblioteca Congreso Nacional de Chile Sitio web: ...] y desde entonces, las diversas radios del país se vieron en la necesidad de adecuar sus parillas de programación para cumplir con la nueva disposición.

\bigskip

Si bien esta propuesta fue promulgada con el fin de promover la diversidad y potenciar la industria musical chilena, se debe considerar que el único ente encargado de distribuir las ganancias a cada artista es la Sociedad Chilena de Derechos de Autor (SCD) que tiene por objetivo administrar derechos autorales generados por la utilización de obras musicales y fonogramas, vale decir, producciones musicales en cualquiera de sus formatos [http://scd.cl/ Sociedad Chilena de Derechos de Autor].

\bigskip

No obstante, muchos de los artistas asociados a la SCD confiesa no saber cómo se realiza la gestión del proceso de derechos de autor sobre sus canciones, es más, según un estudio realizado por Open Business [Open Business. (2013). Desafíos de la gestión colectiva en Chile. 12 de junio, 2016, de Open Business. Sitio web: http://openbusinesslatinamerica.org/2013/04/22/desafios-de-la-gestion-colectiva-en-chile/] “muchas veces el público, los usuarios o los mismos artistas, no saben con precisión en qué consiste el trabajo de esta entidad, especialmente en un país donde ésta posee gran notoriedad pública por temas que no se relacionan directamente con la gestión que ella realiza”. Esto conlleva a generar cierta incertidumbre en los artistas sobre sus ganancias, pues no es factible fiscalizar las más de 30 radioemisoras locales para determinar cuál es la cantidad de reproducciones de sus canciones y llevar un control de la emisión radial de sus obras.

\bigskip

Es por ello, que en este trabajo se realiza un análisis y diseño de una plataforma 
 que permita a los artistas chilenos llevar un control del uso de su música a nivel radial para revisar simultáneamente en diversas radioemisoras online.
 
\lipsum[1-4]
 
La estructura de la presente memoria es la siguiente:

\begin{itemize}
\item El capítulo 1 detalla la propuesta a desarrollar, las razones para hacerlo, y los objetivos que se desean cumplir con ello.

\item El capítulo 2 muestra una indagación en la literatura abarcando principalmente las metodologías que existen para el reconocimiento de canciones digitalizadas, y los servicios y/o algoritmos más conocidos.

\item El capítulo 3 presenta el diseño inicial, explicando la metodología para desarrollar la solución y detalla la forma en que se llevó a cabo la implementación del algoritmo de reconocimiento acústico.

\item En el capítulo 4 se señalan las tareas realizadas para comprobar el rendimiento de la plataforma.

\item El capítulo 5 muestra los resultados obtenidos tras llevar a cabo la etapa anterior, analizando los valores recopilados.

\item Para finalizar, la Concusión resume el trabajo presentado, se comprueba el cumplimiento de los objetivos planteados, y se identifican los posibles trabajos a futuro.
\end{itemize}

% \newpage\null\thispagestyle{empty}\newpage


