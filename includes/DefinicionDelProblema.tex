%!TEX root = ../memoria.tex

\chapter{Definición del problema}

\section{Descripción} \label{sec:Descripcion}

En Chile, la Sociedad de Derechos de Autor es el ente que se encarga de administrar los derechos de propiedad intelectual generados por la utilización de obras y fonogramas musicales, por tanto, su gestión se encarga de determinar los ingresos que los artistas reciben por el uso de sus creaciones musicales. 
El modo de operar se basa en la Ley de Propiedad Intelectual, la cual dispone que todas las radios y canales de televisión deben entregar a la sociedad sus planillas de ejecución, que reúnen la información de cada obra que comunican a través de sus programaciones. 

\bigskip

Por su parte, la SCD señala que la distribución de los derechos se basa en “una muestra aleatoria estadística de aproximadamente 6 días de emisión por cada mes completo enviado por las radioemisoras. Adicionalmente a la muestra aplicada a este rubro de reparto, se incluyen las obras (canciones) difundidas los días 17,18 y 19 de septiembre y 24, 25 y 31 de diciembre de cada año” [ ...], apoyándose además del Software de reconocimiento de música Vericast de BMAT “que tiene como objetivo aumentar la información que es emitida por las radioemisoras que cuenta con señal online ” [...]

\bigskip

Cabe destacar que la licencia para el uso del software Vericast fue adquirida a partir de la convocatoria de licitación pública realizada por el Concejo Nacional de las Artes (CNCA), bajo el programa de Medición Radial, difundido por el gobierno de Chile para el fomento de la música nacional [.../]. De esta forma, la SCD se adjudicó el proyecto, proponiendo la contratación del servicio de Vericast, y la puesta en marcha de su implementación hasta finales del año 2016 [...] y cuyas características se señalan a continuación:


\begin{verbatim}

http://transparenciaactiva.cultura.gob.cl/uploads/marcoNormativo/ebf456ab89212ce3af3173b8f2ebd18c5fc3a123.pdf
PAg 3, las partes convienen… CUATRO]
\end{verbatim}
Y cuyo costo, ascendió a un total de \$290.850.000 pesos chilenos, de los cuales:
\begin{verbatim}[Transcribir: http://transparenciaactiva.cultura.gob.cl/uploads/marcoNormativo/ebf456ab89212ce3af3173b8f2ebd18c5fc3a123.pdf

PAg 3, las partes convienen… TERCERO]
\end{verbatim}

No obstante, el primer balance, tras un año de la promulgación de la ley, reveló ciertos problemas a los que se vieron enfrentados los actores de la industria. Algunos directores de radios establecieron que en algunas ocasiones se programaron artistas y bandas locales que no figuran en los registros de la SCD provocando que no fueran reconocidos por el sistema: “ Si bien hace algunos meses la SCD puso a disposición de las radios una base de datos digitalizada de música nacional, con el fin de complementar catálogos -en especial para las radios regionales-, el problema persiste y con esto muchas veces las cifras del organismo no cuadran con los balances internos de cada señal.   ”[... ]

\bigskip

Los dos elementos que contribuyen a este escenario es que, por un lado, (explicar del encargado de la SCD, estipulado por licitación). Adicionalmente, existe un gran número de artistas que no se encuentran asociados a los registros de las SCD, tales como (buscar y nombrar artistas independientes).

\bigskip

Es en base a esta situación que surge la idea de diseñar y desarrollar una plataforma nacional que permita fiscalizar las más de 30 radioemisoras locales y comprobar de manera automática las parrillas musicales de cada una de éstas, permitiendo también, a los artistas chilenos no asociados a la SCD, llevar un control del uso de su música a nivel radial.


\section{Objetivos}

\subsection{Objetivo general} \label{sec:ObjetivoPrim}
\begin{itemize}
\item Detectar la frecuencia de reproducción de canciones de artistas chilenos en radioemisoras online.
\end{itemize}

\subsection{Objetivos secundarios} \label{sec:ObjetivosSec}
\begin{itemize}
\item Almacenar un catálogo de prueba de canciones nacionales en una base de datos.
\item Diseñar e implementar sistema de reconocimiento acústico para analizar en paralelo un conjunto de radioemisoras online.
\item Implementación de radioemisoras para validar solución desarrollada.
\end{itemize}
\section{Alcances} \label{sec:Alcances}
\lipsum[1-2]